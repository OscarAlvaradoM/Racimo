\documentclass[twocolumn, letterpaper,aps,pra,10pt]{revtex4-1}
\usepackage[spanish]{babel}
\usepackage[utf8]{inputenc}
\usepackage[T1]{fontenc}
\usepackage{times}
\usepackage{calligra}
\usepackage{graphicx}
\usepackage{latexsym}
\usepackage{amsmath,amssymb}
\usepackage{subfigure}
\usepackage{booktabs}
\usepackage{tabulary}
\usepackage{url}
%\usepackage{mhchem}
\spanishdecimal{.}
\usepackage{ragged2e}
\bibliographystyle{unsrt}
\usepackage[usenames,dvipsnames]{pstricks}
\usepackage{epsfig}
\usepackage{pst-grad} % For gradients
\usepackage{pst-plot} % For axes
\usepackage{float}
\usepackage{colortbl}
\usepackage{hyperref}
\usepackage{latexsym}
\usepackage{xcolor}
\usepackage{fancyhdr}
\pagestyle{fancy}

\begin{document}
\renewcommand{\figurename}{{\bf Figura }}
\renewcommand{\tablename}{{\bf Tabla}}
\renewcommand{\thesection}{\arabic{section}}
\renewcommand{\thesubsection}{\arabic{subsection}}

\begin{figure}
\flushleft \includegraphics[width=1in]{unam_logo.jpg}
\end{figure}
\begin{figure}
\flushright \includegraphics[width=1in]{iimas.jpg}
\end{figure}

\lhead{}
\chead{IIMAS, UNAM.}
\rhead{}
\lfoot{Alvarado Morán Óscar Anuar} 
\cfoot{\thepage}
\rfoot{}

\vspace*{-1cm}
\title{Manual para cluster de Raspberry Pi: Racimo.}
\author{Alvarado Morán Óscar Anuar \\Bermúdez Marbán Dante \\Esquivel Flores Oscar \\García Avendaño Martín}
\affiliation{IIMAS \\
Universidad Nacional Autónoma de México}

\maketitle
%%%%%%%%%%%%%%%%%%%%%%%%%%%%%%%%%%%%%%%%%%%%%%%%%%%%%%%%%%%%%%%%%%%%%%%%%%%%%%
Se compraron 5 Raspberry Pi modelo 4 de 4 GB de RAM desde Amazon entrando en el enlace que se muestra a continuacion:\\
\url{https://www.amazon.com.mx/gp/product/B07TLG1HFY/ref=ppx_yo_dt_b_asin_title_o01_s00?ie=UTF8&psc=1} \\

Las tarjetas llegaron en tiempo y forma. Además se compraron 5 micro SD para cargar ahí el sistema operativo y cualquier programa requerido, las memorias se compraron en Amazon entrando en el siguiente enlace: \\
\url{https://www.amazon.com.mx/gp/product/B06XWMQ81P/ref=ppx_yo_dt_b_asin_title_o00_s00?ie=UTF8&psc=1} \\
 
Las memorias se requirieron de una capacidad de 32 GB ya que es el máximo que la tarjeta puede utilizar sin necesidad de cambiar un protocolo de fábrica. Se consideraron tarjetas de este tipo (Clase 10 U3, A1) ya que es el modelo de comunicación más rápida disponible para las Raspberry Pi, las que son A2 no son compatibles por el momento. 

\textbf{Antes de hacer cualquer conexión, cabe recalcar que la conexión a la energía debe ser la última en hacerse.}
Posterior a las compras de los componentes se comenzó la configuración de Software, con lo que evidentemente en primer lugar era necesaria la instalación de un sistema operativo. De inicio se instaló Raspbian siguiendo el siguiente tutorial: \\
\url{https://www.youtube.com/watch?v=Tt2riuPOh6o}
\\

Se utilizó etcher y la página oficial de Raspberry Pi para todo este proceso:
\\
\url{https://www.balena.io/etcher/}\\
\url{https://www.raspberrypi.org/downloads/raspbian/}
\\

Al insertar en la Raspberry Pi la tarjeta microSD que contenía Raspbian no fue necesario hacer alguna otra cosa, ya se mostraba el escritorio con la pantalla que se conectó mediante el HDMI. 

Posteriormente se instaló Julia ya que es el lenguaje de programación con el que se planea trabajar. Se siguió la siguiente guía, pero en realidad es bastante fácil:\\
\url{https://juliaberry.github.io/}
\\
además se instaló jupyter notebook siguiendo partes de la siguiente guía:\\
\url{https://www.raspberrypi.org/blog/julia-language-raspberry-pi/}
\\

Para instalar todo esto se necesitó una conexión a internet. La tarjeta RPi que se tiene trae incluida antena Wi-Fi y BT, además de que incluye un puerto para conectarse a internet mediante cable ethernet, sin embargo \textbf{no se ha podido conectar mediante este método debido a las limitaciones de red que hay en el instituto}.

Por otro lado, se comenzó a probar el funcionamiento de los puertos GPIO que incluye la RPi. Como se pretende usar Julia se probaron mediante una librería llamada PiGPIO que supuestamente es de Julia pero que en verdad resultó siendo la importación de un módulo de Python; se siguió el siguiente tutorial para el código en Julia de un blink con la Raspberry Pi:
\\
\url{https://medium.com/@imkimfung/using-julia-to-control-leds-on-a-raspberry-pi-b320be83e503}
\\

que viene desde un repositorio en GitHub que se muestra a continuación:
\\
\url{https://github.com/JuliaBerry/PiGPIO.jl}
\\

%%%%%%%%%%%%%%%%%%%%%%%%%%%%%%%%%%%%%%%%%%%%%%%%%%%%%%%%%%%%%%%%%%%%%%%%%%%%%%
\begin{thebibliography}{99}
%\bibitem{desc} \url{http://www.physics.csbsju.edu/tk/370/jcalvert/dischg.htm.html}
%\bibitem{pasch} Domínguez, Arturo., \textit{Derivation of the Paschen curve law ALPhA Laboratory Immersion}, 2014 
\end{thebibliography}
\end{document}